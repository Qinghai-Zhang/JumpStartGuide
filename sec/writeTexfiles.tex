\subsection{为什么要注意\LaTeX 源代码的规范}

很多人在写数学文档(包括数学论文和讲义)时只关注编译生成的PDF文件,
而并不关注对应的\LaTeX 源代码,以至于\LaTeX 源代码相当混乱。
这样做看似不影响最终的输出结果,但相比于一份混乱不堪的\LaTeX 源码,
一份符合规范的\LaTeX 源码可以带来不少益处。

首先,规范的\LaTeX 源码可以提升效率。如果是单个人编辑的数学文档,
在需要对文档进行修改时,一个规范的tex代码可以让你更快地定位到需要修改的位置,
省去了一系列繁琐且无意义的定位工作,能显著提升写文档的效率。
在多人合作编辑一篇文档时意义更加明显,
一份规范的tex源代码能让你的合作者更为愉快且高效地阅读你的工作,
从而有助于更好地进行数学上的交流。

其次,规范的\LaTeX 源码可以帮助你更好地审视文档里的语言表达。
在句子结构比较复杂时我们往往需要在源码里比较合适的部分进行换行,
而比较好的换行方式恰恰是依照句子的成分进行换行。
遵循这个规范会帮助我们找到英文写作中的一些问题,
对提升英文写作水平也有相当的帮助。

最后,规范的\LaTeX 源码也可以优化定理证明的书写。
在维持tex源码的规范时,我们往往需要特别关注数学公式的排版。
这有助于我们注意到,在一个长公式中,
哪些部分是重要的,哪些部分相对是比较次要的。
后续审视这段证明时,一份规范的tex源码也可以让我们更容易注意到,
有哪些数学符号有多次复用,
这对我们简化证明的书写过程,
有非常重要的意义。


\subsection{\LaTeX 源码的一些基本规范}

\begin{itemize}
    \item tex文件源码每行的内容量不应超过73列(单个屏幕显示的最大列数)。
    \begin{itemize}
        \item 如果句子过长,则应在句中用回车键换行。
        \item 尽量不要割裂句子中的行内公式。
        \item 长句换行时,比较理想的方式是依照句子的成分进行换行。
        \item 如果是复合句,从句开始部分务必换行。
    \end{itemize}
    \item 每个section或是subsection结束时加两个空行,段落结束加一个空行。
    \item Equation环境中的数学公式同样不应超过73列。
    \item 公式中的断句不应当割裂数学符号,在表达式的连接处断句为最佳。
    \item 特殊环境的说明:
    \begin{itemize}
        \item 运用equation环境输入公式时必须加label,否则用displaymath环境。
        \item 无需引用的公式不要用equation环境,用displaymath即可。
        \item 有重要语义的环境(如Definition, Theorem等)必须加label。
        \item Table/Figure环境也必须加label。
        \item 写label时用大小写字母分开的形式,尽量不要用下划线。
        \item 特殊环境(如Definition, Theorem等)的前后要加一个空行。
        \item 公式的前后是否加空行,根据实际情况而定。
    \end{itemize}
\end{itemize}