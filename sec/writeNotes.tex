
\subsection{为什么要写英文讲义}

荀子说过 “不积跬步,无以至千里;不积小流,无以成江海”,
数学学科的学习也是如此,
数学水平的提升需要长时间的学习与积累,
为了更好地在积累过程中形成自己的“数学知识体系”,
要重点把握以下几个原则。

首先数学的学习需要“少、慢、精、深”,
我们可以每次少学一点,但是一定要确保自己真正理解和掌握了新的知识,
并且将新学的知识与已有的知识联系起来,使得整个体系更加牢固。
就拿盖房子打比方,想要建成属于自己数学知识体系的 “大厦” ,
那就得从底层开始把每一层楼都建得结结实实,让后面楼层的建设能够依赖于前面坚实的基础而越发轻松,
而不是在平地上建一个一个独立的帐篷,这种方式只能建起零碎的小房子,而无法建起 “高楼”。

其次,每过一段时间要对自己的当前的知识体系重新进行一次梳理,
因为每当我们学习了新的知识,我们对旧的知识会有新的理解。
重新梳理的根本目的是让我们的知识体系混乱度最小,
试想如果数学各个学科的知识在我们脑中乱成一锅粥,
这种情况下要我们想起某个知识点就非常困难了,
更别说在实际问题中应用数学了,
所以这种梳理也是回顾和修复知识体系漏洞的好方法。

最后,就是要善于在实践中学习和整理数学。
本科生阶段是通过一门一门的数学课程来学习数学,
而在研究生阶段,数学课程教的那些知识已经完全不够用了,
此时就需要在科研实践中学习和整理数学,
将科研成果、论文中的数学知识整合进数学知识体系中。

道家讲求“道法术”的统一,
撰写自己的英文讲义便是整理数学知识体系之“法”,
通过撰写英文讲义很容易贯彻上面所说的三个原则。
首先,写英文讲义有利于我们主动思考,
有些时候在写讲义的过程中,能够发现许多之前没有注意到的细节,
产生对数学理论更深的理解,
符合“少、慢、精、深”的原则;
其次,英文讲义可视化了自身的知识体系状态,
当遇到问题时,有什么漏洞一目了然,
也便于后期回顾和总结,
符合“定期梳理”的原则;
最后,当在实践过程中学习到新的数学知识,便可以第一时间整理至讲义中,
符合“在实践中学习和整理数学”的原则。
而英文作为当前数学界的通用语言,撰写英文讲义也是为了提升英文表达能力,
为后续撰写英文论文打下坚实的基础。

\subsection{写好一个合格的英文数学讲义}

% 除了遵循前文中数学学习原则的指导外,
平时我们写英文讲义时,也要在细节上严格要求自己,
让讲义能够达到一定的标准,
成为一份“合格”的英文讲义,这里枚举了一些基本的标准:

\begin{itemize}
  \item 保证英文的拼写、语法、用语正确性和恰当性:英文拼写、语句语法的正确性自然是最基本的要求,而用语也要选择数学文章中常用的表达,例如下面两个语句都没有语法错误,但是后者更加符合数学文章的表达习惯。这就需要多阅读文章进行积累,或者在写讲义时多参考别人写的文章。
  \begin{lstlisting}
  M(z) can be decomposed into ...
  M(z) can be considered as the composition of ...
  \end{lstlisting}
  \item 标题格式要统一:文档名、章节标题所有实词首字母大写,例如 \lstinline|Notes on Complex Analysis|;
  二级标题、三级标题的首字母大写,例如 \lstinline|A story of motivations|。
  \item 章节内容的组织要符合数学脉络和读者思路:
    \begin{itemize}
      \item 每章、每节的标题名要准确概括章节的核心思想,尽量用具体的词语而非宽泛的词语:例如想阐述 Mobius 变换保持角的性质时,使用 \lstinline|Preservation of angles| 比 \lstinline|The basic properties| 更为精确。
      \item 讲义内容要完善、有条理:讲义的内容要完善、自成体系,切忌前言不搭后语,以及跳跃性太大,尽量将相似的内容放在一起,便于读者理解。
      例如在引入 Mobius 变换定义时,可以同时引入 singular Mobius 变换、 normalized Mobius 变换的概念,而非一节引入一个概念,这样有利于读者理清多个概念之间的关系。
    \end{itemize}
  \item 定义环境中的“定义主体”要用斜体表示:例如 \lstinline|function| 定义需要使用 \lstinline|\emph{function}|。
  \begin{definition}
    \label{def:function}
    \textnormal{A \emph{function} is a mapping from a set $X$ to a set $Y$.}
  \end{definition}

  % \item 定义、定理、引理、推论环境的选择要合理:数学讲义中环境的选择体现了整个讲义的逻辑,能够帮助读者理清讲义的脉络,在选择环境时思考其与上下文的关系。
  \item 数学公式标点格式要规范:
    \begin{itemize}
      \item 行间数学公式作为描述语句的一部分,则需要加逗号:例如,若数列 $a_n$ 满足
      \begin{equation}
        \label{eq:punctation1}
        \forall \epsilon > 0, \exists N \in \mathbb{N} \text{ s.t. } \forall n \geq N, |a_n - A| < \epsilon,
      \end{equation}
        则称 $a_n$ 的极限为 $A$。这里的行间公式结尾有一个逗号,说明其是描述语句的一部分。
      \item 多行数学公式作为结尾,则前几行加逗号,最后一行加句号:例如
      \begin{align}
        \label{eq:punctation2}
        f_1(x) = 1, \\
        f_2(x) = x.
      \end{align}
    \item 当数学公式作为证明环境结尾时,将证明环境结尾的方框插入公式中:例如
    \begin{proof}
      \begin{displaymath}
        \begin{bmatrix}
          \alpha & -\beta
          \\
          \beta & \alpha
        \end{bmatrix}^{-1}
        =
        \frac{1}{\alpha^2+\beta^2}
        \begin{bmatrix}
          \alpha & \beta
          \\
          -\beta & \alpha
        \end{bmatrix}.
        \qedhere
      \end{displaymath}
    \end{proof}
    从源代码角度即在结尾处增加 \lstinline|\qedhere|。

    \begin{lstlisting}
    \begin{proof}
      \begin{displaymath}
        ...
        \qedhere
      \end{displaymath}
    \end{proof}
    \end{lstlisting}

  \end{itemize}
  \item 为每个定理环境、数学公式提供标签:即在每个 \lstinline|theorem|、\lstinline|equation| 环境中加上 \lstinline|\label{...}|,这样在后续引用时就可以直接使用 \lstinline|\ref{...}|。这样体现了数学知识相互的联系与复用。
\end{itemize}


