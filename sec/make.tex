
什么是 makefile?
或许很多 Winodws 的程序员都不知道这个东西,
因为那些 Windows 的 IDE 都为你做了这个工作,
但作为一个好的专业的程序员,makefile 还是要懂。
特别在 Linux 下的软件编译,你就不能不自己写 makefile 了,
会不会写 makefile,从一个侧面说明了一个人是否具备完成大型工程的能力。
因为,makefile 关系到了整个工程的编译规则。
一个工程中的源文件不计数,其按类型、功能、模块分别放在若干个目录中,
makefile 定义了一系列的规则来指定,
哪些文件需要先编译,哪些文件需要后编译,哪些文件需要重新编译,
甚至于进行更复杂的功能操作,
因为 makefile 就像一个 Shell 脚本一样,其中也可以执行操作系统的命令。
makefile 带来的好处就是——自动化编译,
一旦写好,只需要一个 make 命令,
整个工程完全自动编译,极大的提高了软件开发的效率。


\subsection{程序的编译和链接}

在进入 makefile 之前,
我们不得不先说一下程序编译的一些规范和方法,
一般来说,对于C、C++文件来说,
首先要把源文件(.c/.cpp文件)编译成中间代码文件,
在 Windows 下也就是 .obj 文件,
Ubuntu 下是 .o 文件,即 Object File,这个动作叫做编译。
然后再把大量的 Object File 合成执行文件,这个动作叫作链接。   

编译时,编译器需要检查的是语法,函数与变量的声明是否正确。
对于后者,通常是你需要告诉编译器头文件的所在位置
(头文件中应该只是声明,而定义应该放在C/C++文件中),
只要所有的语法正确,编译器就可以编译出中间目标文件。
一般来说,每个源文件都应该对应于一个中间目标文件。 
链接时,主要是链接函数和全局变量,
所以,我们可以使用这些中间目标文件来链接我们的应用程序。
链接器并不管函数所在的源文件,只管函数的中间目标文件。

由源文件生成可执行程序的基本命令为
\begin{verbatim}
g++ hello.cpp hello1.cpp hello2.cpp -o hello
\end{verbatim}
这里 g++ 是将 gcc 默认语言设为 C++ 的一个特殊的版本
(编译 .c 文件时为 gcc),
链接时它自动使用 C++ 标准库而不用 C 标准库。
-o 指定生成的可执行文件的名称。
然后在命令行中输入程序名可使之运行:
\begin{verbatim}
./hello。
\end{verbatim}

可以使用选项 -c 用来告诉编译器编译源代码但不要执行链接,
输出结果为 .o 文件。
文件默认名与源码文件名相同,只是将其后缀变为 .o。
例如,下面的命令将编译源码文件 hello.cpp 并生成对象文件 hello.o
\begin{verbatim}
g++ -c hellospeak.cpp。
\end{verbatim}

\subsection{Makefile的核心规则}

在正式写一个完整的makefile之前,
我们先来粗略地看一看makefile的规则,
即makefile三要素:
\begin{verbatim}
target : prerequisites
         command
\end{verbatim}

其中 target 可以是一个 object file(目标文件),
也可以是一个执行文件,还可以是一个标签。
\\prerequisites 就是,要生成 target 目标所需要的文件。
command 也就是 make 需要执行的命令。
需要注意的是这里的 command 前是一个 tab 键(不能使用空格)。
这是一个文件的依赖关系,
也就是说,target 这一个或多个的目标文件依赖于 prerequisites 中的文件,
其生成规则定义在 command 中。
说白一点就是说,rerequisites 中如果有一个以上的文件比 target 文件要新的话,
command 所定义的命令就会被执行。
这就是 makefile 的规则,也就是 makefile 中最核心的内容。

那么 makefile 是如何工作的呢?
在默认的方式下,也就是我们只输入 make 命令。
那么,
\begin{itemize}
	\item make 会在当前目录下找名字叫 Makefile 或 makefile 的文件
		(文件命名非常严格,只建议这两种命名方式)。
	\item 如果找到,它会找文件中的第一个目标文件 target),
		并把这个文件作为最终的目标文件。
	\item 如果目标文件不存在(尚未生成),
		或是目标文件所依赖的后面的文件的文件修改时间要比目标文件新,
		那么,他就会执行后面所定义的命令来生成目标文件。
	\item 如果目标所依赖的文件也不存在,
		那么 make 会在当前文件中找目标文件所依赖的文件的依赖性,
		按照上述规则执行。
	\item 最终 make 会一层又一层地去找文件的依赖关系,
		直到编译出第一个目标文件。
\end{itemize}



\subsection{一步步写好 makefile 文件}

我们先来看一个非常简单的例子,假设我们有 4 个文件,
它们分别是 main.cpp,add.cpp,\\ mul.cpp, sub.cpp;
其中 main.cpp 依赖于 add.cpp, mul.cpp, sub.cpp,
为了简单起见,假设 add.cpp,mul.cpp,sub.cpp 之间没有依赖关系。

一个最简单的 makefile 文件就是我们在命令行编译的时候输入的编译代码,
用 makefile 的规范书写就是
\begin{verbatim}
test : main.cpp add.cpp mul.cpp sub.cpp
   g++ -o test main.cpp add.cpp mul.cpp sub.cpp
\end{verbatim}

虽然这个 makefile 文件是可以正常运行的,但如果仅仅是这样,
似乎没有说起来那么神奇。
试想如果我们仅仅修改了 sub.cpp 文件,那么按照 makefile 的规则,
修改后的 sub.cpp 要比 test 新,程序又再次执行了
\begin{verbatim}
g++ -o test main.cpp add.cpp mul.cpp sub.cpp,
\end{verbatim}
这导致了其他不相关的文件 add.cpp,mul.cpp 又要重新编译,
这无疑浪费了很多资源。
基于此,我们可以考虑将编译链接过程分解出来,
即显示的让编译器先生成中间目标文件(.o文件),
修改后的 makefile 文件如下
\begin{verbatim}
test : main.o add.o sub.o mul.o
   g++ -o test main.o add.o sub.o mul.o
main.o : main.cpp
   g++ -c main.cpp
add.o : add.cpp
   g++ -c add.cpp
sub.o : sub.cpp
   g++ -c sub.cpp
mul.o : mul.cpp
   g++ -c mul.cpp
\end{verbatim}

虽然这样的 makefile 基本符合我们的需求,
但是写起来未免有些麻烦,拓展性也不是太强。
比如对于每个源文件,我们都要重新写一次它的编译过程,
即使它们的语法一模一样。
为了解决这个问题,下面我们介绍一些写 makefile 文件常用的技巧
\begin{itemize}
	\item 利用"\# 注释内容"可以添加注释,以此来增加 makefile 的可读性;
	\item 使用变量:在 makefile 中,
		我们可以利用变量来代表某些多处使用而又可能发生变化的内容,
		这样可以节省重复修改的工作,还可以避免遗漏。
		使用规则为
		\begin{verbatim}
		变量名称 = 值列表
		\end{verbatim}
		需要使用的时候可以用 "\$(变量名称)" 来替换即可。
		例如在上述 makefile 中,我们可以使用 ObjFiles 来代替所有的中间目标文件
		\begin{verbatim}
		#(注释) 定义和使用变量ObjFiles
		ObjFiles = main.o add.o sub.o mul.o
		test : $(ObjFiles)
  			 g++ -o test $(ObjFiles)
		...
		\end{verbatim}
	\item 使用函数,在这里我们仅仅介绍两个常用的函数
    \begin{itemize}
    	\item wildcard :扩展通配符;
    		通常用法是使用"\$(wildcard *.cpp)"来获取工作目录下的所有的 .cpp 文件列表。
    	\item patsubst :替换通配符;
    		通常用法是使用"\$(patsubst\%.cpp,\%.o,\$(wildcard *.cpp))"
    		得到在当前目录下可生成的 .o 文件列表。
    		原理为先使用 wildcard 函数获取工作目录下的 .cpp 文件列表,
    		之后将列表中所有文件名的后缀 .cpp 替换为 .o。
    \end{itemize}
    使用函数后的 makefile 文件
	\begin{verbatim}
	# get all .cpp files
	SrcFiles = $(wildcard *.cpp)
	# get all .o files
	ObjFiles = $(patsubst %.cpp,%.o,$(wildcard *.cpp))
	test : $(ObjFiles)
   		g++ -o test $(ObjFiles)
	... 
	\end{verbatim}
	
	\item makefile 的模式匹配规则:
		目标名中需要包含有模式字符 \%(一个),
		包含有模式字符 \% 的目标被用来匹配一个文件名, 
		\% 可以匹配任何非空字符串。
		规则的依赖文件中同样可以使用 \%,
		依赖文件中模式字符 \% 的取值情况由目标中的 \% 来决定。
		最常用的用法是把所有的源文件生成中间目标文件。
		\begin{verbatim}
		%.o : %.cpp
   			g++ -c $< -o $@ 
		\end{verbatim}
		其中规则命令行中使用了自动化变量 \$ <和 \$@。
		自动化变量\$ < 代表规则的依赖,\$@ 代表规则的目标。
		此规则在执行时,命令行中的自动化变量将根据实际的目标和依赖文件取对应值。
		这样一个基本的拓展性强的 makefile 文件初步生成了
		\begin{verbatim}
		# get all .c files
		SrcFiles = $(wildcard *.cpp)
		# get all .o files
		ObjFiles = $(patsubst %.cpp,%.o,$(wildcard *.cpp))
		test : $(ObjFiles)
   			g++ -o test $(ObjFiles)
		%.o : %.cpp
   			g++ -c $< -o $@ 
		\end{verbatim}
	\item 每个 makefile 中都应该写一个清空目标文件(.o和执行文件)的规则,
		这不仅便于重新编译,也很利于保持文件的清洁。
		这是一个基本修养。
		一般的风格都是:
		\begin{verbatim}
		clean: 
   			rm + 需要清理的文件
		\end{verbatim}		
		更为稳健的做法是:
		\begin{verbatim}
		.PHONY : clean
		clean :
   			-@rm -f 需要清理的文件 
		\end{verbatim}	
		其中 .PHONY 意思表示 clean 是一个伪目标(防止产生冲突)。
		而在 rm 命令前面加了一个小减号的意思就是,
		也许某些文件出现问题,但不要管,继续做后面的事。
		@表示不在命令行显示这个命令,
		-f表示强制执行,不管有没有这些文件。
		值得注意的是,clean 的规则不要放在文件的开头,
		不然,这就会变成make的默认目标,相信谁也不愿意这样。
		不成文的规定是——clean 从来都是放在文件的最后。
		有时也可以添加 realclean 进行更深层次的清理。
	\item 添加伪目标 all:因为 makefile 默认第一个目标是"终极目标",
		所以我们约定的做法是使用一个称为"all"的伪目标来作为终极目标,
		它的依赖文件就是那些需要创建的程序。
		\begin{verbatim}
			all : 其他目标
		\end{verbatim}
	\item 添加 run 来运行程序。
	
\end{itemize}

我们展示一下最后的 makefile 文件
\begin{verbatim}
CPP = g++
OFLAG = -o
CFLAG = -c

.PHONY: clean all

all: test

# get all .cpp files
SrcFiles = $(wildcard *.cpp)
# get all .o files
ObjFiles = $(patsubst %.cpp,%.o,$(wildcard *.cpp))

test : $(ObjFiles)
   $(CPP) $(OFLAG) $@ $(ObjFiles)

%.o : %.cpp
   $(CPP) $(CFLAG) $< $(OFLAG) $@ 

run:
   ./test

clean:
   -@rm -f *.o
\end{verbatim}
这样我们只需要在终端输入 make,make run 或者 make clean 
就可以实现程序的编译,运行以及清理工作。






















%%% Local Variables:
%%% mode: latex
%%% TeX-master: "../Guide"
%%% End:
