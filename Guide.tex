\documentclass[a4paper,oneside]{report}
%\documentclass[a4paper]{ctexart}

\usepackage{amssymb}
\usepackage{amsmath}
\usepackage{CJKutf8}
\usepackage{color}
\usepackage{enumerate}
\usepackage{fancyhdr}
\usepackage{geometry}
\usepackage{graphicx}
\usepackage{indentfirst}
\usepackage{latexsym}
\usepackage{mathrsfs}
\usepackage{subfigure}
\usepackage{textcomp}
\usepackage{url}
% \usepackage{unicode-math}
\usepackage{listings}
\usepackage{amsmath}
\usepackage{amsthm}

\usepackage{flafter}
\usepackage{booktabs, longtable}
\usepackage{pxfonts}
\usepackage{cite}

\DeclareMathAlphabet{\mathcal}{OMS}{cmsy}{m}{n}
\let\mathbb\relax % remove the definition by unicode-math
\DeclareMathAlphabet{\mathbb}{U}{msb}{m}{n}

\begin{document}
%\begin{CJK*}{UTF8}{gbsn}
\begin{CJK*}{UTF8}{gkai}
\CJKindent
%------------中文设置--------------------------
\makeatletter %将文献引用作为上标出现,增加括号,
\def\@cite#1#2{\textsuperscript{[{#1\if@tempswa , #2\fi}]}}
\makeatother
\newtheorem{theorem}{{定理}}
\newtheorem{remark}{{注}}
\newtheorem{proposition}[theorem]{{命题}}
\newtheorem{lemma}[theorem]{{引理}}
\newtheorem{corollary}[theorem]{{推论}}
\newtheorem{definition}[theorem]{{定义}}
\newtheorem{question}[theorem]{{问题}}

%\renewcommand{\refname}{\centerline{参考文献}}
\renewcommand{\bibname}{\centerline{参考文献}}
\renewcommand{\tablename}{表}
%\renewcommand{\captionlabeldelim}{\quad}
%===================Image settings========================%
\renewcommand{\figurename}{图}
%\renewcommand{\abstractname}{摘要}
%\renewcommand{\captionlabeldelim}{\quad} %Need caption2 macro package
%===============End image settings========================%
%-----------中文设置--------------------------

		
\title{科研新手快速入门指南}
\author{朱玉可 (\texttt{yukezhu0323@126.com})\\
  张庆海 (\texttt{qinghai@zju.edu.cn})\\
}
\date{浙江大学数学科学学院计算数学系}


\maketitle

%\input{sec/preface}

\pagestyle{empty}

\tableofcontents
\clearpage

\pagestyle{fancy}
\fancyhead{}
\lhead{朱玉可}
\chead{}
\rhead{入门指南}


\chapter{工具篇}

\section{Ubuntu 操作系统及其上软件的安装}
\label{sec:ubuntu}
\input{sec/ubuntuInstall.tex}

\section{强大的文本编辑器---Emacs}
\label{sec:emacs}

Emacs 是一个强大的集成开发环境,
我们可利用它写代码、编译、调试、写文档,
我们甚至可以利用它上网,浏览网页,看视频,收发邮件等,
使用它可以极大地提高我们的工作效率。
学习 Emacs 的关键在于牢记各种快捷键,
减少在键盘和鼠标间的切换。
对于初学者建议按如下步骤学习 Emacs:
\begin{itemize}
	\item 阅读 Emacs Tutorial(进入 Emacs 后使用快捷键 Ctrl-h t)。
		其主要介绍基本的文本查看,编辑和查找操作,
		目的是让初学者对 Emacs 有一个大体的认识。
	\item 如果你想更加深入的了解Emacs,可以阅读书籍 
		Learning GNU Emacs\cite{cameron1996learning}, 
		这里推荐阅读 Chapter 5,8,9。
	\item 此时的你已能解决大多数问题,若还想继续学习 Emacs 的高级功能,
		可以学习 Emacs Lisp 语言,
		参考书籍 An Introduction to Programming in Emacs Lisp
		\cite{chassell2004introduction}。
\end{itemize}
在这里我们不过多介绍 Emacs 中的快捷键,
因为你可以在阅读上面书籍的过程中频繁遇到。
Emacs是绝对值得投入时间和经历的,因为你的回报会远远高出投入。



\subsection{Emacs + AUCTeX}

Emacs + AUCTeX 是一个能极大限度提高 \LaTeX 编辑效率的编辑器。
这是因为其具有强大的快捷键系统,方便的自动补全,
完善的引用系统以及快捷的自定义模板与环境等等。
当然效率的提高程度取决于你对它们的熟悉程度,不要因为刚入手时的困境而放弃。
这里我们简单的介绍一下其使用方法


\begin{itemize}	
	\item \textbf{安装:}可以在终端输入以下命令
	\begin{verbatim}
	sudo apt-get install emacs texlive-full auctex latex-cjk-*
	\end{verbatim}
    当然也可使用 Synaptic Package Manager 进行上述软件的安装。

	\item \textbf{核心操作:}
	\begin{itemize}
  		\item 打开一个文件(C-x C-f + 文件名),该命令后总是跟随这目录/文件名的,
  			如果要打开的文件不存在,就创建新文件;
  		\item 保存文件(C-x C-s),退出Emacs(C-x C-c);
  		\item 编译文档(C-c C-c),预览(C-c C-v),编译文档并查看(C-c C-a)。 
	\end{itemize}

	\item \textbf{基本操作:}
	\begin{itemize}
		\item 插入环境:通过 C-c C-e + 环境类型 命令添加环境
			\begin{verbatim}
			\begin{环境类型}
			\end{环境类型}
			\end{verbatim}
			可以使用 TAB 键查看环境类型列表或者自动补全;
		\item 插入宏:通过 C-c + Enter 命令插入宏,
			接着输入命令名称(usepackage,doucument 等等),
			最后选择输入命令的参数(没有可直接回车)。
			如果需要,你也可以添加相应的标签(可选),
			同样可以使用 TAB 键查看宏类型列表或者自动补全; 
		\item 快速更改字体(这系列命令一致以 C-c C-f 为前缀),如
			\begin{verbatim}
			C-c C-f C-e  // 插入强调字体 \emph{};
			\end{verbatim}
		\item 插入各级标题:通过 C-c C-s 命令插入标题,
			接着输入章节层级(section,subsection 等等),
	    	最后输入标题内容即可;
		\item 进入数学环境:通过 C-c $\sim$ 命令进入数学环境,
			里面快捷键很多,可以在菜单栏查看;
		\item 选择,注释,编译片段文件:
			\begin{verbatim}
			C-c C-r  编译区域
			C-c ;    注释/取消注释区域
			\end{verbatim}
		\item Emacs 内预览(在 Preview 菜单栏内查看);
		\item 多文件管理。
			编写较大的文档时,
			通常会使用 include 或 input 命令将主要章节分离成独立的文件。
			这时就需要在编写子文件的时候,键入编译和预览命令时,
			AUCTeX 能主动定位主文档并执行命令。
			为了做成这件事,需要首先在 .emacs 中添加命令
			\begin{verbatim}
			;; Query for master file.
			(setq-default Tex-master nil)
			\end{verbatim}
			之后,在每次创建新文件的时候我们都需要指定该文件的主文件,
			其基本命令为
			\begin{verbatim}
			C-c _ 
			\end{verbatim}
			然后输入主文件路径信息即可(不显示指明路径即默认该文件为主文件)。
			在编辑子文件时,输入
			\begin{verbatim}
			C-c ^
			\end{verbatim}
			即可返回到主文件。
			在子文件输入编译或者预览命令即可实现全局编译与预览。
	\end{itemize}
\end{itemize}

值得注意的是,使用 AUCTeX 前你不得不学会一些 \LaTeX 的基本句法。
不过作为一个刚接触 \LaTeX 或者接触 \LaTeX 不久的初学者,
设计一个精美的文档是相对困难的,
这时可以适当参考一些优秀的模板(如,\url{http://www.latextemplates.com})。
利用这些模板,我们可以轻松的表达我们想写的内容而不是注重文档设计的形式。
但无论什么时候,都要记住一个原则,即\textbf{保持源码的简洁性}。
大至整个文档的设计,小到每个语法,
其目的是为了方便阅读或者修改。

对于一个科研工作者来说,仅仅会用 \LaTeX 写文档是远远不够的。
在学习的生涯中,你可能会有无数次的报告,展示,答辩等等。
相比于传统的 PPT 展示,
利用 beamer 制作学术讲稿更加方便,美观。
因为其兼容 \LaTeX 命令,对于现有文章中的定义、定理、公式、算法、代码、图表等复杂内容,
只需要简单粘贴到 beamer 里面直接使用即可。
并且其输出为 PDF,具有很强的可读性,跨平台显示也无差异。
下面是一个简单的例子(其效果图见 \ref{fig:beamer}):
\begin{verbatim}
%!TEX program = xelatex

\documentclass{beamer}
\usepackage{xeCJK}

% 使用的主题,Beamer 因为有各种主题才使得有生气
\usetheme{Madrid}

\title[简短的标题]{这是一个简单的 \LaTeX 展示}
\author[作者]{xxx}
\institute[作者机构]{作者机构/信息}
\date[\today]{\today}

\begin{document}

% 标题页帧
\frame{\titlepage}
% 普通帧
\begin{frame}\frametitle{标题}
  这里是需要表达的内容!通常包括基本的文本,表格,图片等等!
\end{frame}

\end{document}
\end{verbatim}

\begin{figure}[htbp]
	\centering
	\subfigure[标题页帧]{
		\includegraphics[width=0.48\linewidth]{png/beamer1}
	}
	\hfill
	\subfigure[普通帧]{
		\includegraphics[width=0.48\linewidth]{png/beamer2}
	}
        \caption{标题和普通帧。}
	\label{fig:beamer}
\end{figure}

这里我们看到的每一页 PDF 即是所谓幻灯片的帧(frame),
Beamer 的帧分为两类,即
\begin{itemize}
	\item 标题页帧,语法为:
	\begin{verbatim}
	\frame{\titlepage}
	\end{verbatim}
	在这上面,一般会有标题、作者、时间、机构,LOGO等信息。
	\item 普通帧,主要为需要展示的内容,基本语法为:
	\begin{verbatim}
	\begin{frame}\frametitle{标题}
  	这里是需要表达的内容!
	\end{frame}
	\end{verbatim}
\end{itemize}

最后我们简要介绍一下 \LaTeX 中的作图工具---PSTricks。
它是一个基于 PS 的宏包,有了它我们就可以直接在 \LaTeX 文档中绘制非常复杂的图形。
但其命令繁琐,不太直观,所以不易熟练掌握。
初期,我们需要知道其基本语法,并学会绘制简单的图形。

\begin{itemize}
	
	\item 基本语法(因为 \LaTeX 绘图指令功能很弱,对较复杂的图形无能为力,
	一般都是用绘图软件事先将图形绘制好,再用图形输入命令插入 \LaTeX 源文件中,
	每一个图形的语法一般如下):
	\begin{verbatim}
	\documentclass[10pt]{article}

	\usepackage{amsmath}
	\usepackage{pstricks,pst-eps}
	\pagestyle{empty}

	\begin{document}
  	  \begin{TeXtoEPS}
    	\begin{pspicture}(5.5,5.5)

      	% Triangle in red:
      	\pspolygon[linecolor=red](1,1)(5,1)(1,4)
      	% Bezier curve in green:
      	\pscurve[linecolor=green,linewidth=2pt,%
      	showpoints=true](5,5)(3,2)(4,4)(2,3)
     	% Circle in blue with radius 1:
      	\pscircle[linecolor=blue,linestyle=dashed](3,2.5){1}

    	\end{pspicture}
  	  \end{TeXtoEPS}
	\end{document}
	\end{verbatim}
	其中紧跟在 pspicture 环境后面的 $(x_0,y_0), (x_1,y_1)$ 参数表示所绘制图形的大小,
	这个图形左下角的坐标在 $(x_0,y_0)$(不显示指明默认为原点),
	右上角的坐标在 $(x_1,y_1)$,生成的图形不能越过这个长方形的范围。
	每一个基本命令包括图形对象和图形参数。
	简单的如点、 线段,复杂的如各种曲线或自定义的图形都被称为图形对象。
	一个图形对象对应着一条命令, 一般的形式是
	\begin{verbatim}
	命令 [选项] ... 
	\end{verbatim}	
	其中选项可以是一个,也可以有多个,多个选项之间用逗号分隔。
	选项中一般设置绘制对象时的线条、颜色属性,这些属性又被称为图形参数。
        使用TeXtoEPS环境可以导出eps格式的矢量图,这种类型的图片不论放大多少倍
        都很清晰并且能够被文字紧密环绕,不会留下大片的空白。
	详细命令可以参考书籍
	PSTricks : PostScript macros for Generic TeX \cite{van1993pstricks}。

       
	\item Makefile 文件(负责由源代码生成 eps 格式图片):
	\begin{verbatim}
	allEps : test.eps
	clean :
   		rm *.aux *.log *.cache
	%.eps : %.dvi
   		dvips $< -E -o $@
	%.dvi : %.tex
   		latex $<
	\end{verbatim}
\end{itemize}

最后只需要在终端输入 make,
即可以得到我们想要绘制的图形(见 \ref{fig:pstricks})。

\begin{figure}[htbp]
	\centering
	\includegraphics[width=0.4\textwidth]{eps/pstricks}
	\caption{使用 PSTricks 绘制的图片。}
	\label{fig:pstricks}
\end{figure}




\section{一次编写,终身受益---Makefile}
\label{sec:make}
\input{sec/make.tex}

\section{软件调试的艺术---GDB}
\label{sec:gdb}
\input{sec/gdb.tex}



\chapter{方法篇}

\section{是什么? 为什么? 怎么做(用)?}
\label{sec:WhatWhyHow}
\input{sec/WhatWhyHow.tex}

\section{算法的契约}
\label{sec:contract}
\input{sec/contract.tex}

\section{更好的算法设计}
\label{sec:designAlgo}
\input{sec/designAlgo.tex}

\section{C++ 中的代码重用---继承和组合}
\label{sec:reuse}
\input{sec/reuse.tex}

\section{英文数学讲义写作原则与示范}
\label{sec:writeNotes}

\subsection{为什么要写英文讲义}

荀子说过 “不积跬步,无以至千里;不积小流,无以成江海”,
数学学科的学习也是如此,
数学水平的提升需要长时间的学习与积累,
为了更好地在积累过程中形成自己的“数学知识体系”,
要重点把握以下几个原则。

首先数学的学习需要“少、慢、精、深”,
我们可以每次少学一点,但是一定要确保自己真正理解和掌握了新的知识,
并且将新学的知识与已有的知识联系起来,使得整个体系更加牢固。
就拿盖房子打比方,想要建成属于自己数学知识体系的 “大厦” ,
那就得从底层开始把每一层楼都建得结结实实,让后面楼层的建设能够依赖于前面坚实的基础而越发轻松,
而不是在平地上建一个一个独立的帐篷,这种方式只能建起零碎的小房子,而无法建起 “高楼”。

其次,每过一段时间要对自己的当前的知识体系重新进行一次梳理,
因为每当我们学习了新的知识,我们对旧的知识会有新的理解。
重新梳理的根本目的是让我们的知识体系混乱度最小,
试想如果数学各个学科的知识在我们脑中乱成一锅粥,
这种情况下要我们想起某个知识点就非常困难了,
更别说在实际问题中应用数学了,
所以这种梳理也是回顾和修复知识体系漏洞的好方法。

最后,就是要善于在实践中学习和整理数学。
本科生阶段是通过一门一门的数学课程来学习数学,
而在研究生阶段,数学课程教的那些知识已经完全不够用了,
此时就需要在科研实践中学习和整理数学,
将科研成果、论文中的数学知识整合进数学知识体系中。

道家讲求“道法术”的统一,
撰写自己的英文讲义便是整理数学知识体系之“法”,
通过撰写英文讲义很容易贯彻上面所说的三个原则。
首先,写英文讲义有利于我们主动思考,
有些时候在写讲义的过程中,能够发现许多之前没有注意到的细节,
产生对数学理论更深的理解,
符合“少、慢、精、深”的原则;
其次,英文讲义可视化了自身的知识体系状态,
当遇到问题时,有什么漏洞一目了然,
也便于后期回顾和总结,
符合“定期梳理”的原则;
最后,当在实践过程中学习到新的数学知识,便可以第一时间整理至讲义中,
符合“在实践中学习和整理数学”的原则。
而英文作为当前数学界的通用语言,撰写英文讲义也是为了提升英文表达能力,
为后续撰写英文论文打下坚实的基础。

\subsection{写好一个合格的英文数学讲义}

% 除了遵循前文中数学学习原则的指导外,
平时我们写英文讲义时,也要在细节上严格要求自己,
让讲义能够达到一定的标准,
成为一份“合格”的英文讲义,这里枚举了一些基本的标准:

\begin{itemize}
  \item 保证英文的拼写、语法、用语正确性和恰当性:英文拼写、语句语法的正确性自然是最基本的要求,而用语也要选择数学文章中常用的表达,例如下面两个语句都没有语法错误,但是后者更加符合数学文章的表达习惯。这就需要多阅读文章进行积累,或者在写讲义时多参考别人写的文章。
  \begin{lstlisting}
  M(z) can be decomposed into ...
  M(z) can be considered as the composition of ...
  \end{lstlisting}
  \item 标题格式要统一:文档名、章节标题所有实词首字母大写,例如 \lstinline|Notes on Complex Analysis|;
  二级标题、三级标题的首字母大写,例如 \lstinline|A story of motivations|。
  \item 章节内容的组织要符合数学脉络和读者思路:
    \begin{itemize}
      \item 每章、每节的标题名要准确概括章节的核心思想,尽量用具体的词语而非宽泛的词语:例如想阐述 Mobius 变换保持角的性质时,使用 \lstinline|Preservation of angles| 比 \lstinline|The basic properties| 更为精确。
      \item 讲义内容要完善、有条理:讲义的内容要完善、自成体系,切忌前言不搭后语,以及跳跃性太大,尽量将相似的内容放在一起,便于读者理解。
      例如在引入 Mobius 变换定义时,可以同时引入 singular Mobius 变换、 normalized Mobius 变换的概念,而非一节引入一个概念,这样有利于读者理清多个概念之间的关系。
    \end{itemize}
  \item 定义环境中的“定义主体”要用斜体表示:例如 \lstinline|function| 定义需要使用 \lstinline|\emph{function}|。
  \begin{definition}
    \label{def:function}
    \textnormal{A \emph{function} is a mapping from a set $X$ to a set $Y$.}
  \end{definition}

  % \item 定义、定理、引理、推论环境的选择要合理:数学讲义中环境的选择体现了整个讲义的逻辑,能够帮助读者理清讲义的脉络,在选择环境时思考其与上下文的关系。
  \item 数学公式标点格式要规范:
    \begin{itemize}
      \item 行间数学公式作为描述语句的一部分,则需要加逗号:例如,若数列 $a_n$ 满足
      \begin{equation}
        \label{eq:punctation1}
        \forall \epsilon > 0, \exists N \in \mathbb{N} \text{ s.t. } \forall n \geq N, |a_n - A| < \epsilon,
      \end{equation}
        则称 $a_n$ 的极限为 $A$。这里的行间公式结尾有一个逗号,说明其是描述语句的一部分。
      \item 多行数学公式作为结尾,则前几行加逗号,最后一行加句号:例如
      \begin{align}
        \label{eq:punctation2}
        f_1(x) = 1, \\
        f_2(x) = x.
      \end{align}
    \item 当数学公式作为证明环境结尾时,将证明环境结尾的方框插入公式中:例如
    \begin{proof}
      \begin{displaymath}
        \begin{bmatrix}
          \alpha & -\beta
          \\
          \beta & \alpha
        \end{bmatrix}^{-1}
        =
        \frac{1}{\alpha^2+\beta^2}
        \begin{bmatrix}
          \alpha & \beta
          \\
          -\beta & \alpha
        \end{bmatrix}.
        \qedhere
      \end{displaymath}
    \end{proof}
    从源代码角度即在结尾处增加 \lstinline|\qedhere|。

    \begin{lstlisting}
    \begin{proof}
      \begin{displaymath}
        ...
        \qedhere
      \end{displaymath}
    \end{proof}
    \end{lstlisting}

  \end{itemize}
  \item 为每个定理环境、数学公式提供标签:即在每个 \lstinline|theorem|、\lstinline|equation| 
  环境中加上 \lstinline|\label{...}|,
  这样在后续引用时就可以直接使用 \lstinline|\ref{...}|。这样体现了数学知识相互的联系与复用。
  \item 如果有需要在数学文档中插入图片,无论是何种格式的图,
  插入的图片大小都需要严丝合缝,\textbf{不能}出现外围的白边。
\end{itemize}

\subsection{参考文献的规范}
我们在写英文数学讲义或是数学论文时,总是需要引用前人的成果,此时就需要我们写一份规范的\texttt{.bib}文件。
由于参考文献格式比较重要,且\texttt{.bib}文件的格式和\texttt{.tex}略有区别,我们另起一节来介绍参考文献的格式规范。
\begin{itemize}
  \item \texttt{.bib}文件的编译:要正常生成参考文献,除了有\texttt{bibtex}源代码外,
  还需要正确的编译过程。只用\texttt{xelatex}编译一遍是无法生成参考文献的。
  正确的编译过程应当是:
  \texttt{xelatex}--\texttt{bibtex}--\texttt{xelatex}--\texttt{xelatex}
  \item 参考文献在\texttt{.bib}文件中的次序:依照作者首字母为第一关键字,
  发表年份为第二关键字,按升序排列。
  \item 对于引用的著作标题,所有的实词首字母都应当大写。
  对于期刊论文标题,标题首字母和人名,专有名词首字母要大写,其余都小写。
  \item 所有引用的著作,一定用@book作为标识符。
  \item 引用著作包含的信息至少需要有title, author, year和publisher。
  \item 引用论文包含的信息至少需要有title, author, journal, pages, year和publisher。
\end{itemize}



\section{\LaTeX 源代码的一些规范}
\label{sec:writeTexfiles}
\subsection{为什么要注意\LaTeX 源代码的规范}

很多人在写数学文档(包括数学论文和讲义)时只关注编译生成的PDF文件,
而并不关注对应的\LaTeX 源代码,以至于\LaTeX 源代码相当混乱。
这样做看似不影响最终的输出结果,但相比于一份混乱不堪的\LaTeX 源码,
一份符合规范的\LaTeX 源码可以带来不少益处。

首先,规范的\LaTeX 源码可以提升效率。如果是单个人编辑的数学文档,
在需要对文档进行修改时,一个规范的tex代码可以让你更快地定位到需要修改的位置,
省去了一系列繁琐且无意义的定位工作,能显著提升写文档的效率。
在多人合作编辑一篇文档时意义更加明显,
一份规范的tex源代码能让你的合作者更为愉快且高效地阅读你的工作,
从而有助于更好地进行数学上的交流。

其次,规范的\LaTeX 源码可以帮助你更好地审视文档里的语言表达。
在句子结构比较复杂时我们往往需要在源码里比较合适的部分进行换行,
而比较好的换行方式恰恰是依照句子的成分进行换行。
遵循这个规范会帮助我们找到英文写作中的一些问题,
对提升英文写作水平也有相当的帮助。

最后,规范的\LaTeX 源码也可以优化定理证明的书写。
在维持tex源码的规范时,我们往往需要特别关注数学公式的排版。
这有助于我们注意到,在一个长公式中,
哪些部分是重要的,哪些部分相对是比较次要的。
后续审视这段证明时,一份规范的tex源码也可以让我们更容易注意到,
有哪些数学符号有多次复用,
这对我们简化证明的书写过程,
有非常重要的意义。


\subsection{\LaTeX 源码的一些基本规范}

\begin{itemize}
    \item tex文件源码每行的内容量不应超过73列(单个屏幕显示的最大列数)。
    \begin{itemize}
        \item 如果句子过长,则应在句中用回车键换行。
        \item 尽量不要割裂句子中的行内公式。
        \item 长句换行时,比较理想的方式是依照句子的成分进行换行。
        \item 如果是复合句,从句开始部分务必换行。
    \end{itemize}
    \item 每个section或是subsection结束时加两个空行,段落结束加一个空行。
    \item Equation环境中的数学公式同样不应超过73列。
    \item 公式中的断句不应当割裂数学符号,在表达式的连接处断句为最佳。
    \item 特殊环境的说明:
    \begin{itemize}
        \item 运用equation环境输入公式时必须加label,否则用displaymath环境。
        \item 无需引用的公式不要用equation环境,用displaymath即可。
        \item 有重要语义的环境(如Definition, Theorem等)必须加label。
        \item Table/Figure环境也必须加label。
        \item 写label时用大小写字母分开的形式,尽量不要用下划线。
        \item 特殊环境(如Definition, Theorem等)的前后要加一个空行。
        \item 公式的前后是否加空行,根据实际情况而定。
    \end{itemize}
\end{itemize}
{\footnotesize
\bibliographystyle{abbrv}
\bibliography{bib/Guide}
}



\end{CJK*}
\end{document}

